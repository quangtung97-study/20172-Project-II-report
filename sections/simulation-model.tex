\documentclass[../report.tex]{subfiles}

\begin{document}
Một mô hình mô phỏng (simulation model) bao gồm một tập các quy tắc định nghĩa cách thức 
mà hệ thống thay đổi theo thời gian, với một trạng thái hiện tại bất kì.
\cite{multi-agent-simulation}
Không nhơ mô hình phân tích, mô hinh mô phỏng không được giải (bằng tính toán hay chứng minh) mà dựa vào 
việc chạy và thay đổi trạng thái của hệ thống và quan sát giá trị tại trạng thái bất kì thời điểm nào. Nó cung cấp 
một cái nhìn vào bên trong hệ thống thay vì chị dựa vào một kết quả cụ thể nào. 
Giả lập không phải công cụ quyết định mà là công cụ hỗ trợ quyết định, cho phép tạo ra những lựa chọn tốt hơn. 

Do sự phức tạp của các bài toán thực tế nên một mô hình mô phỏng chỉ có thể là một mô hình xấp xỉ của hệ thống đích. 
Do đó việc trừu tượng hóa và đơn giản hóa là công việc thiết yếu trong thiết kế một mô hình mô phỏng. Chỉ những đặc trưng 
quan trọng để nghiên cứu và phân tích của hệ thống đích mới nên được đưa vào trong mô hình. 

Có ba loại kĩ thuật mô phỏng thường được sử dụng: 
\begin{itemize}
\item Mô phỏng sự kiện rời rạc - Bao gồm một tập các thực thể được xử lý và tiến triển theo thời gian dựa vào 
sự xuất hiện hay biến mất của các tài nguyên hoặc sự xuất hiện của các sự kiện. Được xây dựng dựa trên 
một hàng đợi các sự kiện. 
\item Mô phỏng hệ thống động - Mô phỏng hệ thống theo thời gian, dựa vào mối quan hệ giữa các thực thể hoặc dựa 
vào một phương trình vi phân nào đó. 
\item Mô phỏng tác tử - Mô phỏng hệ thức ở mức thức. Những cấu trúc ở mức cao được tạo thành bởi các hành động của các 
tác tử và tương tác giữa chúng. 
\end{itemize}

Mặc dù mô phỏng máy tính đã được sử dụng rộng rãi từ những năm 1960 nhưng mô phỏng tác tử chỉ trở nên phổ biến từ những
năm 1990. Nó hiện tại là một công cụ mô hình mô phỏng được xây dựng mạnh mẽ và bắt đầu phổ biến trong giảng dạy và trong 
công nghiệp. Mô phỏng tác tử được sử dụng trong trường hợp mà sự tương tác giữa các tác tử không thể bị xem nhẹ. Nó 
cho phép nghiên cứu về tính động của hệ thống thông qua những đặc điểm của từng cá thể trong môi trường. 
Mô phỏng tác tử là một mô hình hứa hẹn vì nó dựa vào ý tưởng rằng mọi việc của con người được thực hiện nhờ sự thông minh, 
giao tiếp và hợp tác. 
\end{document}

